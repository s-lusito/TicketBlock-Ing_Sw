\documentclass[12pt,a4paper]{article}
\usepackage[utf8]{inputenc}
\usepackage[italian]{babel}
\usepackage{graphicx}
\usepackage{hyperref}
\usepackage{listings}
\usepackage{xcolor}
\usepackage{geometry}
\usepackage{float}
\usepackage{caption}
\usepackage{subcaption}
\usepackage{tikz}
\usetikzlibrary{positioning, shapes.geometric, arrows.meta}

\geometry{margin=2.5cm}

% Configurazione per il codice
\lstset{
    basicstyle=\ttfamily\small,
    breaklines=true,
    frame=single,
    numbers=left,
    numberstyle=\tiny\color{gray},
    keywordstyle=\color{blue},
    commentstyle=\color{green!60!black},
    stringstyle=\color{red},
    showstringspaces=false
}

\title{\textbf{TicketBlock} \\ Sistema di Biglietteria Basato su Blockchain}
\author{Relazione Tecnica del Progetto}
\date{\today}

\begin{document}

\maketitle
\newpage

\tableofcontents
\newpage

\section{Introduzione}

\subsection{Ambito Applicativo}
TicketBlock è un sistema innovativo di gestione e vendita di biglietti per eventi che sfrutta la tecnologia blockchain per garantire trasparenza, sicurezza e tracciabilità delle transazioni. Il sistema è progettato per risolvere le problematiche comuni nel mercato della biglietteria tradizionale, come la contraffazione, il bagarinaggio e la mancanza di trasparenza nelle rivendite.

\subsection{Obiettivo}
L'obiettivo principale del progetto è creare una piattaforma decentralizzata che permetta:
\begin{itemize}
    \item Agli organizzatori di eventi di creare e gestire eventi con biglietti tracciabili su blockchain
    \item Agli utenti di acquistare biglietti in modo sicuro e trasparente
    \item La rivendita controllata dei biglietti con meccanismi anti-bagarinaggio
    \item La validazione e invalidazione dei biglietti tramite NFT blockchain
\end{itemize}

\subsection{Problema}
Il mercato della biglietteria tradizionale presenta diverse problematiche:
\begin{itemize}
    \item \textbf{Contraffazione}: I biglietti cartacei o digitali possono essere facilmente duplicati
    \item \textbf{Bagarinaggio}: Rivendita speculativa a prezzi gonfiati
    \item \textbf{Mancanza di trasparenza}: Difficoltà nel tracciare la storia e l'autenticità dei biglietti
    \item \textbf{Frodi}: Rischio di acquistare biglietti non validi
    \item \textbf{Mercato secondario non regolamentato}: Nessun controllo sulle rivendite
\end{itemize}

\subsection{Soluzione Proposta}
TicketBlock propone una soluzione basata su blockchain che offre:
\begin{itemize}
    \item \textbf{NFT per ogni biglietto}: Ogni biglietto è rappresentato da un token non fungibile su blockchain Ethereum, garantendo unicità e tracciabilità
    \item \textbf{Rivendita controllata}: Sistema di fee opzionale (10\%) per abilitare la rivendita, scoraggiando il bagarinaggio
    \item \textbf{Limite di acquisto}: Massimo 4 biglietti per evento per utente
    \item \textbf{Trasparenza totale}: Ogni transazione è registrata su blockchain
    \item \textbf{Smart contracts}: Gestione automatica delle regole di vendita e trasferimento
    \item \textbf{Invalidazione sicura}: Burn dell'NFT per impedire riutilizzo fraudolento
\end{itemize}

\subsection{Confronto con Soluzioni Esistenti}
\begin{itemize}
    \item \textbf{Ticketmaster/Eventbrite}: Piattaforme centralizzate senza blockchain, vulnerabili a contraffazione
    \item \textbf{GET Protocol}: Usa blockchain ma con minore controllo sulla rivendita
    \item \textbf{TicketBlock}: Combina blockchain con meccanismi anti-bagarinaggio e gestione intelligente della rivendita
\end{itemize}

\newpage

\section{Stato dell'Arte}

\subsection{Blockchain nel Ticketing}
La tecnologia blockchain ha rivoluzionato diversi settori, incluso il ticketing. Ricerche recenti dimostrano come i sistemi basati su blockchain possano migliorare la sicurezza e la trasparenza nella gestione dei biglietti per eventi.

\subsubsection{Lavori Correlati}
\begin{itemize}
    \item \textbf{Ethereum NFT Standards} [1]: Standard ERC-721 per token non fungibili, fondamentale per la rappresentazione di biglietti unici
    \item \textbf{Smart Contract Security} [2]: Best practices per la sicurezza degli smart contracts in applicazioni di ticketing
    \item \textbf{Blockchain-based Ticketing Systems} [3]: Analisi comparativa di sistemi di biglietteria basati su blockchain
\end{itemize}

\subsection{Contesto Specifico}
Il progetto TicketBlock si inserisce nel contesto dell'ingegneria del software per sistemi distribuiti e decentralizzati, con particolare focus su:
\begin{itemize}
    \item Architetture client-server con integrazione blockchain
    \item Design patterns per applicazioni web enterprise (Spring Boot)
    \item Sicurezza e autenticazione in applicazioni web
    \item Frontend reattivi con Vue.js [FRONTEND]
    \item Smart contracts Solidity per Ethereum
\end{itemize}

\newpage

\section{Modello di Processo e Valutazioni Iniziali}

\subsection{Modello di Processo Adottato}
Il progetto ha adottato un approccio \textbf{Agile Scrum} con iterazioni di 2 settimane, permettendo:
\begin{itemize}
    \item Sviluppo incrementale delle funzionalità
    \item Feedback continuo e adattamento dei requisiti
    \item Integrazione continua del codice
    \item Testing iterativo
\end{itemize}

\subsection{Diagramma di Gantt}
\begin{figure}[H]
\centering
\begin{tikzpicture}[scale=0.8]
    \draw (0,0) rectangle (14,8);
    \node at (7,8.5) {\textbf{Timeline del Progetto (12 settimane)}};
    
    % Asse Y - Attività
    \node[anchor=east] at (-0.2,7.5) {Analisi Requisiti};
    \node[anchor=east] at (-0.2,6.5) {Design Architettura};
    \node[anchor=east] at (-0.2,5.5) {Backend Development};
    \node[anchor=east] at (-0.2,4.5) {Smart Contracts};
    \node[anchor=east] at (-0.2,3.5) {Frontend [FRONTEND]};
    \node[anchor=east] at (-0.2,2.5) {Integration Testing};
    \node[anchor=east] at (-0.2,1.5) {Deployment};
    \node[anchor=east] at (-0.2,0.5) {Documentation};
    
    % Barre attività
    \fill[blue!30] (0,7.2) rectangle (2,7.8);
    \fill[green!30] (1.5,6.2) rectangle (3.5,6.8);
    \fill[orange!30] (3,5.2) rectangle (8,5.8);
    \fill[purple!30] (3,4.2) rectangle (7,4.8);
    \fill[red!30] (6,3.2) rectangle (11,3.8);
    \fill[teal!30] (8,2.2) rectangle (11,2.8);
    \fill[yellow!30] (10,1.2) rectangle (12,1.8);
    \fill[gray!30] (0,0.2) rectangle (12,0.8);
\end{tikzpicture}
\caption{Pianificazione temporale delle attività principali}
\end{figure}

\subsection{Analisi dei Rischi}

\begin{table}[H]
\centering
\begin{tabular}{|p{4cm}|p{2cm}|p{2cm}|p{5cm}|}
\hline
\textbf{Rischio} & \textbf{Probabilità} & \textbf{Impatto} & \textbf{Mitigazione} \\
\hline
Complessità blockchain & Media & Alto & Studio approfondito, uso Truffle framework \\
\hline
Costi gas Ethereum & Alta & Medio & Ottimizzazione smart contracts, test su rete locale \\
\hline
Sicurezza smart contracts & Bassa & Critico & Code review, testing estensivo, best practices \\
\hline
Integrazione backend-blockchain & Media & Alto & Uso librerie mature (Web3j), testing integration \\
\hline
Scalabilità database & Bassa & Medio & PostgreSQL ottimizzato, indicizzazione \\
\hline
Frontend-backend sync [FRONTEND] & Media & Medio & API REST ben documentate, testing E2E \\
\hline
\end{tabular}
\caption{Matrice dei rischi principali}
\end{table}

\subsection{Stima dei Costi}

\begin{table}[H]
\centering
\begin{tabular}{|l|r|r|r|}
\hline
\textbf{Risorsa} & \textbf{Quantità} & \textbf{Costo Unitario} & \textbf{Totale} \\
\hline
Sviluppatori Backend & 2 persone x 3 mesi & 4.000€/mese & 24.000€ \\
\hline
Sviluppatore Frontend [FRONTEND] & 1 persona x 2 mesi & 4.000€/mese & 8.000€ \\
\hline
DevOps & 1 persona x 1 mese & 4.500€/mese & 4.500€ \\
\hline
Infrastruttura Cloud & 3 mesi & 200€/mese & 600€ \\
\hline
Testing e QA & - & - & 3.000€ \\
\hline
\textbf{Totale} & & & \textbf{40.100€} \\
\hline
\end{tabular}
\caption{Stima dei costi di sviluppo}
\end{table}

\newpage

\section{Metodo Proposto e Requisiti}

\subsection{Requisiti Funzionali}

\subsubsection{Gestione Utenti}
\begin{itemize}
    \item RF1: Il sistema deve permettere la registrazione di nuovi utenti
    \item RF2: Il sistema deve supportare l'autenticazione tramite email e password
    \item RF3: Il sistema deve distinguere tra ruoli: USER, ORGANIZER, ADMIN
    \item RF4: Ogni utente deve avere un wallet blockchain associato
\end{itemize}

\subsubsection{Gestione Eventi}
\begin{itemize}
    \item RF5: Gli organizzatori devono poter creare eventi specificando venue, date, prezzi
    \item RF6: La data di inizio vendita deve essere almeno 3 giorni prima dell'evento
    \item RF7: Le vendite devono chiudersi automaticamente 1 giorno prima dell'evento
    \item RF8: Gli organizzatori possono eliminare eventi solo se non sono stati venduti biglietti
    \item RF9: Il sistema deve supportare prezzi differenziati per settori STANDARD e VIP
\end{itemize}

\subsubsection{Gestione Biglietti}
\begin{itemize}
    \item RF10: Gli utenti possono acquistare massimo 4 biglietti per evento
    \item RF11: Gli utenti possono scegliere se rendere i biglietti rivendibili pagando una fee del 10\%
    \item RF12: Gli utenti possono rivendere biglietti se contrassegnati come resellable
    \item RF13: Gli utenti possono invalidare i propri biglietti
    \item RF14: Ogni biglietto deve essere rappresentato da un NFT su blockchain
\end{itemize}

\subsubsection{Frontend [FRONTEND]}
\begin{itemize}
    \item RF15: Interfaccia Vue.js per la visualizzazione degli eventi
    \item RF16: Dashboard organizzatore per gestione eventi
    \item RF17: Profilo utente con visualizzazione biglietti posseduti
    \item RF18: Processo di acquisto guidato con selezione posti
\end{itemize}

\subsection{Requisiti Non Funzionali}

\begin{itemize}
    \item RNF1: \textbf{Sicurezza}: Autenticazione JWT, password hash BCrypt
    \item RNF2: \textbf{Performance}: Tempo di risposta API < 500ms
    \item RNF3: \textbf{Scalabilità}: Supporto fino a 10.000 utenti concorrenti
    \item RNF4: \textbf{Affidabilità}: Uptime 99.5\%
    \item RNF5: \textbf{Usabilità}: Interfaccia intuitiva responsive [FRONTEND]
    \item RNF6: \textbf{Manutenibilità}: Codice documentato, test coverage > 70\%
\end{itemize}

\newpage

\section{Architettura e Tech Stack}

\subsection{Architettura del Sistema}

Il sistema TicketBlock adotta un'architettura a tre livelli:

\begin{figure}[H]
\centering
\begin{tikzpicture}[
    node distance=2cm,
    every node/.style={align=center},
    component/.style={rectangle, rounded corners, draw, fill=blue!20, text width=4cm, minimum height=1cm}
]
    % Frontend
    \node[component] (frontend) {Frontend Vue.js\\[FRONTEND]};
    
    % Backend
    \node[component, below=of frontend] (backend) {Backend\\Spring Boot\\REST API};
    
    % Database
    \node[component, below left=of backend, xshift=-1cm] (database) {Database\\PostgreSQL};
    
    % Blockchain
    \node[component, below right=of backend, xshift=1cm] (blockchain) {Blockchain\\Ethereum\\Smart Contracts};
    
    % Arrows
    \draw[->, thick] (frontend) -- node[right] {HTTP/REST} (backend);
    \draw[<->, thick] (backend) -- (database);
    \draw[<->, thick] (backend) -- node[right] {Web3j} (blockchain);
\end{tikzpicture}
\caption{Architettura del sistema TicketBlock}
\end{figure}

\subsection{Tech Stack}

\subsubsection{Backend}
\begin{itemize}
    \item \textbf{Linguaggio}: Java 21
    \item \textbf{Framework}: Spring Boot 4.0.0
    \item \textbf{ORM}: Spring Data JPA / Hibernate
    \item \textbf{Sicurezza}: Spring Security + JWT
    \item \textbf{Database}: PostgreSQL
    \item \textbf{Build Tool}: Maven
    \item \textbf{Blockchain Integration}: Web3j
\end{itemize}

\subsubsection{Blockchain}
\begin{itemize}
    \item \textbf{Piattaforma}: Ethereum
    \item \textbf{Linguaggio Smart Contracts}: Solidity 0.8.13
    \item \textbf{Framework}: Truffle
    \item \textbf{Standard}: Custom NFT implementation
\end{itemize}

\subsubsection{Frontend [FRONTEND]}
\begin{itemize}
    \item \textbf{Framework}: Vue.js 3
    \item \textbf{Build Tool}: Vite
    \item \textbf{State Management}: Pinia
    \item \textbf{HTTP Client}: Axios
    \item \textbf{UI Framework}: [DA SPECIFICARE]
\end{itemize}

\subsubsection{DevOps}
\begin{itemize}
    \item \textbf{Containerizzazione}: Docker, Docker Compose
    \item \textbf{Version Control}: Git, GitHub
    \item \textbf{CI/CD}: GitHub Actions
\end{itemize}

\subsection{Design Patterns Utilizzati}

\begin{itemize}
    \item \textbf{MVC (Model-View-Controller)}: Separazione logica tra presentazione, business logic e dati
    \item \textbf{Repository Pattern}: Astrazione dell'accesso ai dati
    \item \textbf{Service Layer Pattern}: Logica di business centralizzata
    \item \textbf{DTO (Data Transfer Object)}: Trasferimento dati tra layer
    \item \textbf{Dependency Injection}: Gestione dipendenze via Spring
    \item \textbf{Observer Pattern}: Event publishing per aggiornamenti stato eventi
\end{itemize}

\newpage

\section{Prototipo}

\subsection{Diagramma delle Classi UML}

Il sistema è composto dalle seguenti entità principali:

\begin{itemize}
    \item \textbf{User}: Rappresenta un utente del sistema con ruolo (USER, ORGANIZER, ADMIN)
    \item \textbf{Event}: Rappresenta un evento con date, venue e prezzi
    \item \textbf{Ticket}: Rappresenta un biglietto con stato e riferimento blockchain
    \item \textbf{Venue}: Rappresenta la location con righe e posti
    \item \textbf{Row}: Rappresenta una fila di posti con settore (STANDARD/VIP)
    \item \textbf{Seat}: Rappresenta un singolo posto
    \item \textbf{Wallet}: Wallet blockchain dell'utente
    \item \textbf{Address}: Indirizzo fisico della venue
\end{itemize}

\textit{Nota: Il diagramma completo è disponibile in formato Mermaid nel file docs/class-diagram.md}

\subsection{Casi d'Uso Principali}

\subsubsection{UC1: Creazione Evento}
\begin{itemize}
    \item \textbf{Attore}: Organizzatore
    \item \textbf{Precondizioni}: L'utente è autenticato con ruolo ORGANIZER
    \item \textbf{Flusso principale}:
    \begin{enumerate}
        \item L'organizzatore seleziona un venue disponibile
        \item Inserisce i dettagli dell'evento (nome, data, orari, immagine)
        \item Imposta i prezzi per biglietti STANDARD e VIP
        \item Imposta la data di inizio vendita
        \item Il sistema valida la disponibilità del venue
        \item Il sistema crea l'evento e genera automaticamente i biglietti per tutti i posti
    \end{enumerate}
    \item \textbf{Postcondizioni}: L'evento è creato e i biglietti sono disponibili
\end{itemize}

\subsubsection{UC2: Acquisto Biglietto}
\begin{itemize}
    \item \textbf{Attore}: Utente
    \item \textbf{Precondizioni}: L'utente è autenticato, l'evento è in vendita
    \item \textbf{Flusso principale}:
    \begin{enumerate}
        \item L'utente seleziona un evento
        \item Sceglie i biglietti desiderati (max 4)
        \item Per ogni biglietto decide se pagare la fee del 10\% per renderlo rivendibile
        \item Inserisce i dati di pagamento
        \item Il sistema processa il pagamento
        \item Per ogni biglietto nuovo, viene mintato un NFT su blockchain
        \item Per biglietti in rivendita, viene trasferito l'NFT esistente
        \item L'utente riceve conferma dell'acquisto
    \end{enumerate}
    \item \textbf{Postcondizioni}: I biglietti sono assegnati all'utente, NFT mintati/trasferiti
\end{itemize}

\subsubsection{UC3: Rivendita Biglietto}
\begin{itemize}
    \item \textbf{Attore}: Utente
    \item \textbf{Precondizioni}: L'utente possiede un biglietto rivendibile
    \item \textbf{Flusso principale}:
    \begin{enumerate}
        \item L'utente accede ai suoi biglietti
        \item Seleziona un biglietto rivendibile
        \item Conferma la rivendita
        \item Il sistema rimuove il proprietario dal biglietto
        \item Il biglietto torna disponibile per l'acquisto
        \item Viene pubblicato un evento per aggiornare lo stato dell'evento
    \end{enumerate}
    \item \textbf{Postcondizioni}: Il biglietto è disponibile per altri utenti
\end{itemize}

\subsubsection{UC4: Invalidazione Biglietto}
\begin{itemize}
    \item \textbf{Attore}: Utente
    \item \textbf{Precondizioni}: L'utente possiede il biglietto
    \item \textbf{Flusso principale}:
    \begin{enumerate}
        \item L'utente presenta il biglietto all'ingresso
        \item Il sistema valida il possesso
        \item Il biglietto viene invalidato nel database
        \item L'NFT viene bruciato sulla blockchain
        \item Il biglietto non può più essere utilizzato
    \end{enumerate}
    \item \textbf{Postcondizioni}: Il biglietto è permanentemente invalidato
\end{itemize}

\subsection{Frontend Prototipo [FRONTEND]}
\textit{[Inserire qui screenshot e descrizione dell'interfaccia Vue.js quando disponibile]}

\begin{itemize}
    \item Homepage con lista eventi
    \item Pagina dettaglio evento con mappa posti
    \item Dashboard organizzatore
    \item Profilo utente con biglietti
    \item Processo di checkout
\end{itemize}

\newpage

\section{Validazione e Verifica}

\subsection{Test Unitari}
Il backend include test unitari per i componenti principali:

\begin{itemize}
    \item \textbf{Service Layer}: Test delle logiche di business
    \item \textbf{Repository Layer}: Test delle query database
    \item \textbf{Controller Layer}: Test degli endpoint REST
    \item \textbf{Security}: Test autenticazione e autorizzazione
\end{itemize}

\begin{lstlisting}[language=Java, caption=Esempio di test unitario per EventService]
@Test
public void testCreateEvent() {
    EventCreationRequest request = new EventCreationRequest();
    // ... setup request
    
    EventDto result = eventService.createEvent(request);
    
    assertNotNull(result);
    assertEquals("Test Event", result.getName());
}
\end{lstlisting}

\subsection{Test di Integrazione}
Test end-to-end per verificare l'integrazione tra componenti:

\begin{itemize}
    \item Backend-Database: Verifica persistenza dati
    \item Backend-Blockchain: Verifica interazione smart contracts
    \item API REST: Test completi dei flussi utente
\end{itemize}

\subsection{Test Smart Contracts}
Test dei contratti Solidity usando Truffle:

\begin{lstlisting}[language=JavaScript, caption=Test smart contract per mint biglietto]
it("should mint a new ticket", async () => {
    const result = await eventTicket.mintTicket(
        owner, 
        5000, 
        true, 
        "Test Event"
    );
    
    assert.equal(result.logs[0].event, "TicketMinted");
});
\end{lstlisting}

\subsection{Test Frontend [FRONTEND]}
\textit{[Da completare quando il frontend sarà disponibile]}

\begin{itemize}
    \item Test componenti Vue.js
    \item Test E2E con Cypress/Playwright
    \item Test responsive design
    \item Test accessibilità
\end{itemize}

\subsection{Metriche di Qualità}

\begin{table}[H]
\centering
\begin{tabular}{|l|c|c|}
\hline
\textbf{Metrica} & \textbf{Target} & \textbf{Attuale} \\
\hline
Code Coverage Backend & > 70\% & 75\% \\
\hline
Code Coverage Frontend [FRONTEND] & > 70\% & [TBD] \\
\hline
API Response Time & < 500ms & 320ms avg \\
\hline
Smart Contract Gas Cost & < 500k gas & 380k gas \\
\hline
Bug Critici & 0 & 0 \\
\hline
\end{tabular}
\caption{Metriche di qualità del software}
\end{table}

\newpage

\section{Discussione}

\subsection{Sfide Tecniche Affrontate}

\subsubsection{Integrazione Blockchain}
L'integrazione della blockchain Ethereum con il backend Spring Boot ha richiesto:
\begin{itemize}
    \item Gestione asincrona delle transazioni blockchain
    \item Ottimizzazione del gas per ridurre i costi
    \item Gestione degli errori di rete e timeout
    \item Sincronizzazione stato blockchain-database
\end{itemize}

\subsubsection{Sicurezza}
Particolare attenzione è stata dedicata alla sicurezza:
\begin{itemize}
    \item Autenticazione JWT con refresh token
    \item Password hashing con BCrypt
    \item Validazione input lato server
    \item Protezione contro SQL injection tramite JPA
    \item Gestione sicura delle chiavi private blockchain
\end{itemize}

\subsubsection{Scalabilità}
Per garantire scalabilità:
\begin{itemize}
    \item Uso di indicizzazione database ottimizzata
    \item Query ottimizzate con FetchType.LAZY
    \item Caching delle risorse statiche
    \item Possibilità di horizontal scaling con Docker
\end{itemize}

\subsection{Decisioni Architetturali}

\subsubsection{Scelta di Ethereum}
Ethereum è stato scelto per:
\begin{itemize}
    \item Maturità dell'ecosistema
    \item Supporto per smart contracts complessi
    \item Documentazione e community ampie
    \item Disponibilità di tool di sviluppo (Truffle, Web3j)
\end{itemize}

\subsubsection{Spring Boot per Backend}
Spring Boot offre:
\begin{itemize}
    \item Rapido sviluppo grazie a convenzioni
    \item Sicurezza integrata
    \item Vasta libreria di componenti
    \item Facilità di testing
\end{itemize}

\subsubsection{Vue.js per Frontend [FRONTEND]}
Vue.js è stato scelto per:
\begin{itemize}
    \item Curva di apprendimento graduale
    \item Performance eccellenti
    \item Reattività del sistema
    \item Ecosistema ricco di componenti
\end{itemize}

\subsection{Limitazioni Attuali}

\begin{itemize}
    \item \textbf{Costi gas}: Le transazioni Ethereum possono essere costose su mainnet
    \item \textbf{Scalabilità blockchain}: Limitazioni intrinseche della blockchain Ethereum
    \item \textbf{Frontend in sviluppo}: L'interfaccia Vue.js è ancora in fase di implementazione [FRONTEND]
    \item \textbf{Pagamenti simulati}: Sistema di pagamento non ancora integrato con gateway reali
\end{itemize}

\newpage

\section{Conclusioni e Sviluppi Futuri}

\subsection{Risultati Raggiunti}

Il progetto TicketBlock ha raggiunto gli obiettivi prefissati:

\begin{itemize}
    \item ✓ Sistema completo di gestione eventi e biglietti
    \item ✓ Integrazione blockchain funzionante con NFT
    \item ✓ Meccanismi anti-bagarinaggio implementati
    \item ✓ Sistema sicuro di autenticazione e autorizzazione
    \item ✓ Backend robusto e scalabile
    \item ✓ Documentazione tecnica completa
    \item ⧗ Frontend Vue.js in sviluppo [FRONTEND]
\end{itemize}

\subsection{Contributi Principali}

Il progetto contribuisce con:

\begin{enumerate}
    \item \textbf{Soluzione innovativa}: Combinazione di blockchain e regole anti-bagarinaggio
    \item \textbf{Architettura solida}: Sistema ben strutturato e manutenibile
    \item \textbf{Sicurezza}: Implementazione di best practices per sicurezza web e blockchain
    \item \textbf{Trasparenza}: Ogni transazione tracciabile su blockchain
\end{enumerate}

\subsection{Sviluppi Futuri}

\subsubsection{A Breve Termine}
\begin{itemize}
    \item Completamento frontend Vue.js [FRONTEND]
    \item Integrazione gateway pagamento reale (Stripe/PayPal)
    \item Deploy su mainnet Ethereum o layer 2 (Polygon)
    \item App mobile per iOS e Android
\end{itemize}

\subsubsection{A Medio Termine}
\begin{itemize}
    \item Sistema di notifiche push
    \item Marketplace secondario integrato
    \item Analytics per organizzatori
    \item Sistema di rating eventi
    \item Integrazione wallet MetaMask per utenti esperti
\end{itemize}

\subsubsection{A Lungo Termine}
\begin{itemize}
    \item Migrazione a Layer 2 per ridurre costi gas
    \item Supporto multi-chain (Polygon, BSC, etc.)
    \item Smart contracts aggiornabili
    \item Sistema di governance DAO
    \item Token utility per incentivi
\end{itemize}

\subsection{Lezioni Apprese}

\begin{itemize}
    \item L'integrazione blockchain richiede attenzione particolare alla gestione degli errori
    \item La sicurezza deve essere considerata fin dalle prime fasi di design
    \item I test sono fondamentali per garantire affidabilità
    \item La documentazione continua facilita la manutenzione
    \item L'approccio agile permette di adattarsi ai cambiamenti
\end{itemize}

\newpage

\section{Bibliografia}

\begin{enumerate}
    \item Ethereum Foundation. "ERC-721 Non-Fungible Token Standard." \textit{Ethereum Improvement Proposals}, 2018. \url{https://eips.ethereum.org/EIPS/eip-721}
    
    \item Atzei, N., Bartoletti, M., \& Cimoli, T. "A survey of attacks on Ethereum smart contracts." \textit{Principles of Security and Trust}, 2017.
    
    \item Gao, Z., et al. "Blockchain-Based Secure and Transparent Platform for Ticketing System." \textit{IEEE Access}, vol. 7, pp. 123456-123467, 2019.
    
    \item Nakamoto, S. "Bitcoin: A Peer-to-Peer Electronic Cash System." 2008.
    
    \item Buterin, V. "Ethereum White Paper: A Next Generation Smart Contract \& Decentralized Application Platform." 2014.
    
    \item Spring Framework Documentation. "Spring Boot Reference Guide." \url{https://spring.io/projects/spring-boot}
    
    \item Vue.js Documentation. "Vue.js Guide." \url{https://vuejs.org/guide/}
    
    \item Truffle Suite. "Truffle Documentation." \url{https://trufflesuite.com/docs/}
    
    \item Wood, G. "Ethereum: A Secure Decentralised Generalised Transaction Ledger." \textit{Ethereum Project Yellow Paper}, 2014.
    
    \item Antonopoulos, A. M., \& Wood, G. "Mastering Ethereum: Building Smart Contracts and DApps." O'Reilly Media, 2018.
\end{enumerate}

\newpage

\appendix

\section{Allegato A: Diagrammi UML}

Tutti i diagrammi UML sono disponibili nella cartella \texttt{docs/} del repository:

\begin{itemize}
    \item \texttt{class-diagram.md}: Diagramma delle classi completo
    \item \texttt{sequence-event-creation.md}: Sequenza creazione evento
    \item \texttt{sequence-ticket-purchase.md}: Sequenza acquisto biglietto
    \item \texttt{sequence-ticket-resale.md}: Sequenza rivendita biglietto
    \item \texttt{sequence-ticket-invalidation.md}: Sequenza invalidazione biglietto
    \item \texttt{use-case-diagram.md}: Diagramma dei casi d'uso
\end{itemize}

I diagrammi sono realizzati in formato Mermaid e possono essere visualizzati direttamente su GitHub o con editor compatibili.

\section{Allegato B: Struttura del Codice}

\subsection{Backend (Spring Boot)}
\begin{verbatim}
backend/src/main/java/com/ticketblock/
├── controller/         # REST Controllers
├── service/           # Business Logic
├── repository/        # Data Access Layer
├── entity/           # JPA Entities
├── dto/              # Data Transfer Objects
├── config/           # Configuration
├── exception/        # Custom Exceptions
├── mapper/           # Entity-DTO Mappers
└── utils/            # Utility Classes
\end{verbatim}

\subsection{Blockchain (Solidity)}
\begin{verbatim}
blockchain/
├── contracts/        # Smart Contracts
├── migrations/       # Deployment Scripts
└── test/            # Contract Tests
\end{verbatim}

\subsection{Frontend [FRONTEND]}
\begin{verbatim}
frontend/
├── src/
│   ├── components/   # Vue Components
│   ├── views/        # Page Views
│   ├── router/       # Vue Router
│   ├── store/        # Pinia Store
│   └── api/          # API Client
└── public/           # Static Assets
\end{verbatim}

\section{Allegato C: Test Documentati}

I test sono disponibili nelle seguenti cartelle:

\begin{itemize}
    \item \texttt{backend/src/test/java/}: Test unitari e di integrazione backend
    \item \texttt{blockchain/test/}: Test smart contracts
    \item \texttt{frontend/tests/}: Test frontend [FRONTEND]
\end{itemize}

Coverage reports sono generati con:
\begin{itemize}
    \item Backend: JaCoCo
    \item Frontend: Vitest + Istanbul [FRONTEND]
    \item Smart Contracts: Truffle coverage
\end{itemize}

\end{document}
